\documentclass[10pt,conference,twocolumn]{article}
\usepackage{graphicx} % Required for inserting images

\title{Artigo sobre Antivírus - EP II RC}
\author{Armando Augusto Marchini Vidal,\\
Marcos Vilela Rezende Júnior}
\date{Dezembro 2023}

\begin{document}
\maketitle

\section* {1. Introdução}

\subsection*{A importância dos dados na era digital.}
Na era digital, a relevância dos dados vai além da mera comodidade. Eles são o pilar fundamental para a eficiência e inovação em diversos setores da sociedade. No contexto empresarial, a análise de dados tornou-se uma peça-chave na formulação de decisões estratégicas. As empresas dependem da coleta e interpretação eficiente de dados para entender melhor o mercado, antecipar tendências, otimizar operações e criar estratégias competitivas.\\
Por outro lado, os indivíduos também passaram a depositar sua confiança em plataformas digitais para documentar suas atividades diárias. Seja por meio de redes sociais, aplicativos de saúde ou serviços de armazenamento em nuvem, as pessoas geram e compartilham uma quantidade massiva de dados pessoais online. Esses dados podem incluir desde preferências de compra até informações sensíveis de saúde e localização.\\
A conectividade global oferece uma acessibilidade incrível, mas ao mesmo tempo intensifica os riscos associados ao manuseio dos dados. A perda ou comprometimento de dados sensíveis não apenas ameaça a privacidade individual, mas também pode acarretar consequências financeiras e sociais devastadoras. A exposição indevida de informações pessoais pode levar a fraudes financeiras, roubo de identidade e outros crimes cibernéticos.\\
A sociedade tem se tornado cada vez mais dependente da tecnologia digital, e com isso a necessidade de proteção e segurança dos dados torna-se imperativa. Empresas e indivíduos precisam implementar medidas rigorosas de segurança cibernética para mitigar os riscos associados à coleta, armazenamento e transmissão de dados.

\subsection*{ Crescente ameaça cibernética e necessidade de proteção.}
O constante avanço das tecnologias digitais traz consigo não apenas inovações benéficas, mas também uma escalada nas ameaças cibernéticas. Os ataques, cada vez mais sofisticados, variam desde esquemas de phishing astutos até intrusões em larga escala, como o ransomware. Essas investidas não apenas comprometem dados cruciais, mas também têm o potencial de paralisar operações empresariais, resultando em danos financeiros significativos e impactos irreversíveis nas reputações das organizações afetadas.\\
Em um contexto de crescente interconectividade digital, a proteção cibernética transcende a mera precaução, tornando-se uma necessidade imperativa. À medida que a tecnologia continua a evoluir, a adaptação constante e a inovação em segurança cibernética são indispensáveis para enfrentar e superar as ameaças em constante evolução que permeiam o ambiente digital contemporâneo.

\subsection*{ O papel dos antivírus na segurança digital.}
Os antivírus constituem a primeira linha de defesa na guerra contra ameaças digitais. Essas ferramentas especializadas utilizam algoritmos sofisticados e bancos de dados em constante atualização para reconhecer padrões de comportamento malicioso. Ao realizar varreduras contínuas em sistemas e arquivos em busca de sinais de ameaças, os antivírus desempenham um papel crucial na identificação e remoção de vírus, malware e outras formas de software indesejado. Além disso, muitos antivírus agora incorporam recursos de proteção em tempo real, oferecendo uma camada adicional de segurança contra ataques em constante evolução. Em um cenário digital cada vez mais complexo e dinâmico, a presença e eficácia dessas ferramentas são fundamentais para garantir a integridade e a segurança dos sistemas digitais.

\section* {2. Fundamentos dos Antivírus}

\subsection*{ O que são antivírus e como funcionam.}
Os antivírus desempenham um papel crucial na salvaguarda da integridade e segurança dos sistemas computacionais, sendo concebidos com a finalidade específica de identificar, prevenir e erradicar software malicioso, como vírus, worms, trojans e spywares, que representam ameaças potenciais à confidencialidade e integridade dos dados. O funcionamento dessas ferramentas é intrinsecamente vinculado à capacidade de detecção de padrões característicos dessas ameaças, fazendo uso de uma variedade de métodos e técnicas.

A análise heurística, por exemplo, constitui uma abordagem que capacita os antivírus a identificarem comportamentos suspeitos, mesmo na ausência de uma exposição prévia a uma ameaça específica. Além disso, a busca por assinaturas digitais associadas a códigos maliciosos representa uma técnica convencional e eficaz para a identificação e neutralização de ameaças já conhecidas.

\subsection*{Evolução dos Antivírus ao Longo do Tempo}

Ao longo das últimas décadas, testemunhamos uma evolução substancial no campo dos antivírus, impulsionada pela necessidade de adaptação contínua diante de ameaças em constante mutação. Inicialmente concebidos de forma mais simplificada, esses programas centravam-se primariamente na detecção de vírus reconhecidos, sendo atualizados por meio de assinaturas. No entanto, face à crescente complexidade das ameaças, os antivírus evoluíram consideravelmente, incorporando tecnologias avançadas, tais como análise comportamental, sandboxing e aprendizado de máquina. A presença generalizada de inteligência artificial e aprendizado profundo representa, atualmente, características comuns em antivírus modernos, potencializando uma detecção mais eficiente e ágil de ameaças emergentes.

\subsection*{Antivírus Tradicionais versus Soluções Abrangentes}

Os antivírus tradicionais, embora desempenhem um papel fundamental, frequentemente concentram-se exclusivamente na identificação e remoção de malware. Em contrapartida, as soluções de segurança mais abrangentes expandem significativamente seu escopo, visando lidar com uma gama mais ampla de ameaças. Para além da detecção de malware, essas soluções incorporam funcionalidades como firewalls para monitoramento e controle do tráfego de rede, proteção contra phishing para prevenir ataques de engenharia social, controle parental destinado a resguardar os membros mais jovens da família e implementação de criptografia para assegurar a confidencialidade dos dados. A integração de tecnologias avançadas, como machine learning e inteligência artificial, confere a essas soluções uma abordagem proativa na identificação e mitigação de ameaças de natureza complexa. Em suma, a constante evolução dessas ferramentas reflete a busca incessante por salvaguardar a segurança cibernética em um cenário digital dinâmico e desafiador.


\section* {3. Ameaças Cibernéticas Principais}

\subsection*{As principais ameaças combatidas pelos antivírus.}
Os antivírus desempenham um papel crucial na defesa contra uma miríade de ameaças cibernéticas que evoluem constantemente. Ao longo dos anos, as técnicas dos cibercriminosos tornaram-se mais sofisticadas, exigindo que os programas antivírus evoluam para acompanhar essa evolução. Conversamos com ChatGPT e o Bing para obtermos as principais ameaças combatidas pelos antivírus. O GPT nos proporcionou uma lista mais robusta, da qual destacamos:
\begin{itemize}
\item
Malware: Vírus, worms, trojans e spyware continuam a ser ameaças persistentes. Malwares são programas maliciosos que buscam explorar vulnerabilidades em sistemas para roubo de dados, interrupção de operações e até mesmo para controle remoto de dispositivos.
\item 
Ransomware: Uma ameaça particularmente devastadora, o ransomware criptografa os arquivos de uma vítima, exigindo o pagamento de um resgate em troca da chave de descriptografia. Essa forma de ataque tem causado prejuízos significativos a empresas e indivíduos.
\item 
Phishing: Táticas de engenharia social, como e-mails fraudulentos, sites falsos e mensagens de texto enganosas, são usadas para obter informações confidenciais, como senhas e dados bancários, explorando a confiança das vítimas.
\item 
Ataques de dia zero: Explorando vulnerabilidades previamente desconhecidas em softwares, os ataques de dia zero podem ocorrer antes que os desenvolvedores tenham tido a oportunidade de criar uma correção. Os antivírus desempenham um papel crucial na detecção precoce desses ataques.
\item
Botnets: Redes de dispositivos infectados que são controlados remotamente por hackers, muitas vezes para realizar ataques distribuídos de negação de serviço (DDoS) ou para distribuição de spam.
\end{itemize}

\subsection*{ Estatísticas Recentes sobre Ataques Cibernéticos.}
O cenário de ameaças cibernéticas tem evoluído rapidamente, com um aumento constante no número e na sofisticação dos ataques. Novamente fomos a nossas IAs para consultar informações. Podemos dizer que as estatísticas recentes revelam uma realidade alarmante para a segurança das redes de computadores:
\begin{itemize}
\item
Golpes no WhatsApp no Brasil:
Mais de 3 milhões de golpes foram registrados apenas no aplicativo de mensagens WhatsApp no Brasil. Isso demonstra como os criminosos estão explorando plataformas populares para atingir um grande número de usuários.
\item 
Ransomware em Ascensão:
Em 2021, os ataques de ransomware (que envolvem o sequestro de dados e a exigência de resgate) totalizaram US\$102,3 milhões por mês. Esses ataques afetaram empresas, organizações governamentais e até mesmo indivíduos, causando prejuízos significativos.
\item 
Aumento dos Ataques Cibernéticos no Brasil:
No primeiro semestre de 2022, houve 31,5 bilhões de tentativas de ataques cibernéticos direcionadas a empresas no Brasil. Esse número representa um aumento alarmante de 94\% em relação ao ano anterior. As organizações precisam estar preparadas para enfrentar essas ameaças constantes.
No segundo trimestre de 2022, os ataques cibernéticos no Brasil aumentaram 46\% em comparação com o mesmo período do ano anterior. Isso destaca a urgência de medidas preventivas e de proteção digital.
\item 
Ataques Cibernéticos em 2023:
Somente em 2023, o Brasil já sofreu 23 bilhões de ataques cibernéticos. Entre eles, os golpes de vírus bancários tiveram um crescimento de 20\%. Esses ataques podem resultar em roubo de informações financeiras e comprometimento da privacidade dos usuários.
\end{itemize}

Em resposta a essas ameaças em constante evolução, a comunidade de segurança cibernética continua a desenvolver e aprimorar estratégias de defesa, incluindo a implementação de soluções antivírus avançadas, conscientização do usuário e práticas de segurança robustas em redes de computadores.

\section* {4. Funcionalidades dos Antivírus}
Dentre as funcionalidades de um antivírus podemos citar: Análise em tempo real, Detecção heurística, Atualizações automáticas de definições, Proteção contra phishing e ameaças online e Firewall integrado. Para entrar em mais detalhes, pedimos para que o Copilot do Bing e o ChatGPT selecionassem as mais importantes. Como resultado temos, respectivamente, Análise em tempo real e Atualizações automáticas de definições. Esses tópicos foram escolhidos com base na eficácia de aplicação e preparação de um antivírus.
\subsection*{Análise em tempo real}
O antivírus monitora constantemente as atividades do sistema em execução, identificando e respondendo ameaças. Temos como características:
\begin{itemize}
\item
Monitoramento contínuo: Todos os arquivos, processos e atividades do sistema são verificados.
\item
Heurística e assinaturas: Combinação de técnicas como análise de comportamento com definições conhecidas de ameaças para identificar padrões.
\item
Verificação de arquivos em acesso: Verificação do conteúdo de arquivos abertos, executados, copiados ou modificados.
\item
Proteção contra exploits: Monitoramento de atividades que indicam tentativas de explorar vulnerabilidades do sistema.
\item
Bloqueio em tempo real: Ao detectar uma ameaça, o antivírus pode bloquear, quarentenar ou remover o arquivo/processo malicioso.
\item
Atualizações automáticas: Definições de vírus e outras ameaças são mantidas atualizadas automaticamente pelo antivírus.
\item
Mínimo impacto no desempenho: Antivírus projetado para ter o mínimo impacto no desempenho e nos processos do computador do usuário.
\end{itemize}

\subsection*{Atualizações automáticas de definições}
O antivírus se mantém atualizado para as informações mais recentes sobre as ameaças cibernéticas por esse processo. Ele garante a eficácia do antivírus e possui os seguintes aspectos:
\begin{itemize}
\item
Definições de vírus e malware: Bancos de dados com informações para identificar ameaças (assinatura de código, comportamentos, entre outros) para combatê-las.
\item
Atualizações regulares e automáticas: Essas atualizações ocorrem regularmente e garantem que o antivírus esteja sempre atualizado com informações sobre ameaças que surgem a todo momento.
\item
Reações a ameaças emergentes: Com o surgimento de novas ameaças, fornecedores de antivírus devem desenvolver definições para identificação delas.
\item
Assinaturas digitais e técnicas de detecção: Assinaturas digitais de malware conhecido, técnicas de detecção heurística e comportamentais e outras estratégias estão incluídas nas definições de cada atualização.
\item
Automatização de processo: A atualização automática, sem interferência do usuário, é essencial para que o antivírus verifique a disponibilidade de atualizações e baixe-as.
\item
Conexão com a internet: Essas atualizações automáticas requerem conexão com a internet para permitir atualizações para definições mais recentes.
\end{itemize}



\section* {5. Impacto dos Antivírus na Performance}
\subsection*{Formas de Impacto}
Diversos fatores podem ser observados em um antivírus que impactam direta ou indiretamente no desempenho de um sistema. Alguns dos pontos principais que devem ser observados são:

\begin{itemize}
\item
A Variação de Desempenho:
É preciso notar qual o tamanho do impacto de cada antivírus na performance de um sistema. Enquanto alguns antivírus são focados no baixo impacto, outros consomem mais recursos para uma maior proteção. Testes feitos por terceiros podem ser encontrados online.
\item
Os Recursos do Sistema:
Os recursos consumidos pelo antivírus podem incluir a CPU, a memória RAM e o armazenamento. A complexidade de realização de suas funções de verificação e proteção determinam o quanto de cada um desses recursos será consumido.
\item
As Configurações do Antivírus:
Alguns antivírus podem oferecer opções de configuração que permitem ao usuário ajustar o equilíbrio entre segurança e desempenho.
\item
Varreduras Agendadas:
Uma das formas de contornar o problema de consumo de recursos (e por consequência o impacto no desempenho) é através das varreduras agendadas. Elas podem ser feitas em momentos de baixa atividade do usuário para competir menos pelos recursos do sistema. De forma contrária, se estas varreduras forem feitas em momentos de alto uso por parte do usuário, o consumo e a competição pelos recursos será maior.
\item
Atualizações e Verificações em Tempo Real:
Atualizações em tempo real, assim como verificações constantes de arquivos em execução, tem um impacto direto no desempenho, especialmente para sistemas antigos e/ou com recursos limitados.
\item
Hardware e Tecnologia:
De forma geral, como visto pelo item anterior, sistemas mais antigos e com maior limitação de recursos irão sofrer mais com o uso de antivírus, tendo um impacto de desempenho muito mais visível.
\end{itemize}

É possível ressaltar alguns antivírus que possuem impacto em desempenho, apesar de baixo. Por exemplo: o F-Secure (suportado apenas pelo Windows) venceu prêmios de Melhor Proteção e Melhor Desempenho da AV-TEST em diversos anos e pode ser considerado como uma das melhores opções; e também o Panda Dome Antivirus que possui alta eficiência em proteção e uma execução na nuvem, que otimiza o desempenho.

Por outro lado, também é possível notar impactos positivos no desempenho desses sistemas com a utilização de antivírus. Apresentando soluções como por exemplo a otimização de tarefas ou limpeza de memória, os computadores podem iniciar maisr apidamente, funcionar sem muitos problemas e, obviamente, remover vírus que impactam seu desempenho, mesmo que estes tenham infectado o sistema antes da instalação do antivírus.

\subsection*{Discussão sobre o equilíbrio entre proteção e desempenho.}
Para encontrar um equilíbrio entre o impacto no desempenho do sistema e a proteção garantida pelo antivírus, oa principais fatores a serem levados em consideração são:
\begin{itemize}
\item
Necessidades Individuais:
A priorização do usuário entre a segurança do sistema e seu desempenho. Isso é um fator que varia de pessoa a pessoa.
\item
Tipo de Uso do Computador: 
Tarefas intensivas que demandam muito do sistema podem exigir um antivírus com impacto mínimo.
\end{itemize}

Esses fatores devem ser considerados na escolha de um público-alvo para a empresa dona do antivírus, visto que o escopo pode variar muito e, com isso, a necessidade e a priorização.\newline

Dentre as formas de evitar esses problemas, temos (entre soluções que podem ser aplicadas de forma geral):
\begin{itemize}
\item
Configurações Ajustáveis: 
Para conseguir abranger um número maior de necessidades, uma alta personalização de configurações permite que o usuário decida a relação entre proteção/impacto de forma mais livre.
\item
Atualizações Incrementais: 
Baixando apenas novas definições de ameaça, ao invés de todo o banco de dados, um antivírus pode diminuir seu impacto enquanto garante proteção contra ameaças desconhecidas.
\end{itemize}

\section*{6. Estudo de Caso}
Dentre os testes de desempenho de sistemas com e sem antivírus, podemos destacar e analisar o teste de performance feito pela AV-Comparatives. Esse teste avalia o impacto do software antivírus no desempenho do sistema.
Em 2015, ano que será analisado, os testes foram realizados em uma máquina com um processador Intel Core i7, 8GB de RAM e discos SSD, sob um sistema Windows 10 Home 64-Bit atualizado. Os produtos de segurança foram avaliados com as configurações padrão e com uma conexão ativa à Internet. Foram realizadas as seguintes atividades/testes: cópia de arquivos, arquivamento/desarquivamento, instalação de aplicativos, lançamento de aplicativos, download de arquivos, navegação na web e PC Mark 10 Professional Testing Suite.
\subsection* {Comparação de diferentes soluções antivírus no mercado.}
Tendo como objeto de estudo o antivírus F-Secure, temos os seguintes resultados:
\begin{itemize}
\item
Cópia de arquivos: 0,9 segundos a mais do que o sistema sem antivírus (média de 11,8 segundos)
\item
Arquivamento/desarquivamento: 0,8 segundos a mais do que o sistema sem antivírus (média de 23,8 segundos)
\item
Instalação de aplicativos: 0,7 segundos a mais do que o sistema sem antivírus (média de 29,4 segundos)
\item
Lançamento de aplicativos: 0,3 segundos a mais do que o sistema sem antivírus (média de 6,6 segundos)
\item
Download de arquivos: 0,1 segundos a mais do que o sistema sem antivírus (média de 10,1 segundos)
\item
Navegação na web: 0,1 segundos a mais do que o sistema sem antivírus (média de 10,1 segundos)
\item
PC Mark 10: 0,1\% a menos do que o sistema sem antivírus (média de 100\%)
\end{itemize}
O F-Secure (no ano de 2015 em que a análise foi feita) ficou em segundo lugar nas comparações entre os 19 softwares, perdendo apenas para o ESET Smart Security 9.0 que teve uma interferência praticamente nula no sistema.
Dentre as medidas usadas, o F-Secure se destacou principalmente nas atividades de arquivos (cópias, arquivamento/desarquivamento) e instalação de aplicativos, tendo um impacto menor que a maioria de seus concorrentes.
O F-Secure, nessa análise, se mostrou um dos produtos mais leves e eficientes do mercado, com um baixíssimo impacto de desempenho no Windows 10. Isso o torna uma ótima opção para a busca de antivírus e o mostra como um excelente exemplo de eficiência no que foi estudado ao longo do artigo.

\section* {7. Conclusão}

\subsection*{Recapitulação do papel dos antivírus na proteção de dados.}
Ao longo desse artigo, foi demostrada o impacto que os antivírus tem na preservação e confidencialidade dos dados no mundo digital. São ferramentas que desempenham um papel proativo na detecção e neutralização de ameaças cibernéticas, fortalecendo a segurança dos dispositivos. Nesse cenário em constante evolução , é de extrema importância reconhecer que os antivírus são essenciais para a garantia da resiliência e robustez das defesas contra potenciais ataques, de forma a consolidar um ambiente digital mais seguro para seu usuário.

\subsection*{Relação de custos e benefícios de se ter um antivírus.}
A partir do estudo sobre o aumento de ataques cibernéticos, as funcionalidades de um antivírus e seu impacto no desempenho de um sistema, resta refletir se de fato vale a pena contratar um software de antivírus para proteção. \newline
Após analisar todas as informações coletadas ao longo do artigo, conclui-se que sim, de fato o uso de antivírus se mostra muito útil mesmo nos dias de hoje. O aumento de ataques cibernéticos constitui o principal motivo, pois traz a necessidade de uma proteção extra ao sistema do computador, seja ele pessoal ou não. Entendendo as funcionalidades dos antivírus, vemos que eles oferecem a proteção ideal para esses ataques.\newline
Por fim, temos como fator de análise o impacto de um antivírus no desempenho do sistema do computador. A partir do resultado de estudos de terceiros, fica claro que, de forma geral, o impacto de um antivírus é praticamente nulo, e não deve ser prejudicial ao usuário.


\section* {8. Referências}

https://www.cnnbrasil.com.br/tecnologia/levantamento-mostra-que-ataques-ciberneticos-no-brasil-cresceram-94/ \\

Ransomware em 2021: dados, principais ataques e grupos mais ativos (welivesecurity.com)\\

Clonagem do WhatsApp faz três milhões de vítimas no Brasil em 2020; veja como se proteger - Pequenas Empresas Grandes Negócios | Mundo digital (globo.com)\\

Brasil teve 23 bilhões de tentativas de ataques cibernéticos (olhardigital.com.br)\\

https://olhardigital.com.br/2023/09/11/seguranca/qual-e-o-melhor-antivirus-para-o-seu-pc/\\

https://www.tecmundo.com.br/seguranca/266405-mito-verdade-antivirus-deixar-meu-pc-rapido.htm\\

https://www.pcrisk.pt/top-antivirus/11261-f-secure-anti-virus\\

https://www.pandasecurity.com/pt/\\

https://www.eset.com/br/artigos/antivirus-mais-leve/\\

https://www.eset.com/br/artigos/como-escolher-um-antivirus/\\

https://summitsaude.estadao.com.br/tecnologia/avancos-tecnologicos-impulsionados-pela-pandemia/\\

https://www.tecmundo.com.br/antivirus/82992-descubra-antivirus-impacta-desempenho-computador.htm\\

https://pplware.sapo.pt/software/afinal-qual-o-antivirus-que-mais-afecta-a-performance-do-pc/\\

https://pt.safetydetectives.com/best-antivirus/f-secure/

https://www.av-comparatives.org/tests/performance-test-october-2015/

\end{document}